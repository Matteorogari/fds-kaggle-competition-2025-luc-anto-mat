\section{Feature Engineering and Modeling Architecture}

The methodology focused on quantitatively representing the battle's final state using an optimized \textbf{Stacking Ensemble} architecture.

\subsection{Feature Engineering}
Features were designed based on core game mechanics (condition, power, type interactions) and grouped as follows:

\subsubsection{Core Features (10 and 12 Sets)}
These features encode \textbf{attrition} and \textbf{static potential}:
\begin{itemize}
    \item Delta Mean HP and Delta Survivors: Direct measures of residual health and fighting reserve.
    \item Total Stat Differences: Deltas in summed base statistics ($\Delta \text{Atk}$, $\Delta \text{Spe}$, $\Delta \text{HP}$) to model raw static power advantage.
\end{itemize}

\subsubsection{Strategic Features (17 Set Only)}
These features capture complex, dynamic aspects essential for predictive reliability:
\begin{itemize}
    \item \textbf{Meta Type Advantage Metric:} Calculated based on cross-damage multipliers, recognizing the type system as the most critical $matchup$ mechanic.
    \item \textbf{Weighted Active Leader Power:} An index prioritizing \textbf{Speed (Spe)} to reflect the strategic value of initiative.
\end{itemize}

\subsection{Stacking Ensemble Architecture}
All methodologies adopted a \textbf{Stacking Ensemble} architecture , combining the stability of linear models with the non-linearity of tree-based models.

\subsubsection{Tier-0 Base Learners}
The model set was diversified to ensure the complementarity of Out-Of-Fold (OOF) predictions:
\begin{itemize}
    \item \textbf{Logistic Regression (LR):} Provides low-bias, high-stability predictions, ensuring low correlation with tree models.
    \item \textbf{Boosting Models (XGBoost, HGBT):} Capture non-linear interactions, offering high accuracy.
    \item \textbf{Random Forest and KNN:} Included for ensemble (RF) and non-parametric (KNN) data perspectives.
\end{itemize}
Hyperparameter calibration used $\texttt{GridSearchCV}$ or $\texttt{RandomizedSearchCV}$, optimizing the models based on the \textbf{Log Loss} metric.
